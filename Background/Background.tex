\chapter{Background and Related Work}\label{background}


\section{Previous Work}\label{background_previous_work}

    %Talk about everything. Even the stuff I didn't use
    %The AP data from Arsham

    In the first year of this project much of the initial research was done. In this section some of the key topics will be briefly discussed.

    \subsection{Air Quality}\label{background_air_quality}

Initial research discovered that a definition of air quality is a difficult problem which has not yet been solved. Many institutions measure and record what they refer to as ``air quality'', however the only thing these measurements have in common is that no two are the same. In order to research the area of ``air quality'' further we must have a concrete definition of what air quality is. Current definitions give a simple definition of ``air quality'' as being antonymous to that of air pollution in that higher air quality is the equivalent to low air pollution~\cite{bcaq}. This implies that not only should the metrics used for defining air quality be considered, but also those used to define air pollution.

Looking at statements and measurements by various authorities around the world, such as those seen in \cite{epapollutants,airqualityobjectives,cleanairnavigation,naaqs,whoguidelines}, we can build a list of common pollutants which are measured:


\begin{itemize}
\item Sulphur dioxide
\item Nitrogen dioxide
\item PM10 
\item PM2.5
\end{itemize}

These chemicals can form the basis of pollutants which in turn can help define ``air quality''. By looking at other sources~\cite{meaningsofenvironmentalterms},  proposed definition of ``air quality'' is as follows:

\begin{quote}
``The current measurements of the concentrations of certain pollutants in the air relative to the requirements of one or more biotic species or to any human need or purpose.''
\end{quote}


    \subsection{Geostatistical Mapping}
    
        It was discovered that when working with a sparse actual data set, obtained by using a sensor built by a previous years student, \emph{Eric Staples}, work would have to be performed in order to make use of the data effectively. Generally, heat maps are used to show various concentrations and levels of variables in a two dimensional environment and so the same would be used in this project. The interpolation method which should be used was a more difficult problem. Initial research suggested that some form of convolution filter, or blur, when applied to a sufficiently dense data set would give a reasonable approximation, however this was later found to be incorrect, and would not apply to a sparse data set likely to be generated from any experiments. Further research showed that when interpolating environmental data such as pollutant concentrations, the most common method is kriging, a statistical method which fits a curve to the data set. Kriging is used as the basis for many advanced models~\cite{regressionkriging}, including LUR.

    \subsection{Compressive Sampling}

        Compressive sampling was looked into briefly as a method of enhancing our data collection methodology. The basis behind compressive sampling is that more measurements are taken in areas of high variance, with less measurements taken in areas of lower variance~\cite{compressivesampling}. Essentially this is a variable method of conforming to the Nyquist theorem, which is discussed further in section~\ref{prediction_evaluation_motivation}. The work on this area was discovered to involve much background research into the nature of the pollutants and the areas they build up and would not work if there were a change to this environment. Furthermore it is outwith our control when using fixed schedule vehicles, such as public transport buses, and so this technique was not used in the final solution.

    \subsection{Sensor Creation}

        Initially this project was intended to have a practical component, however external circumstances restricted this and the physical implementation was cancelled. Before this change, research was done into building a physical sensor using cheap, off the shelf parts, which would be dynamically calibrated using the techniques discussed by D. Hasenfratz et al. in~\cite{ontheflycalibration}. Individual sensors were researched and purchased in order to make a prototype, before the physical implementation was cancelled. 

    \subsection{Current Work Context}

        With the above work taking place last year, what remains is the work completed this year. This work includes the following:

        \begin{itemize}
            \item Researched different networking simulators.
            \item Created a simulation of buses moving around Edinburgh using traces of actual bus movements, taking readings and sending these back to a central server through WiFi access points, using the Omnet++ simulator.
            \item Optimised this simulation in terms of speed and data output size.
            \item Created, implemented and tested an opportunistic forwarding algorithm.
            \item Ran tests on various parameter changes in the simulation and aggregated and analysed the results.
            \item Acquired and normalised data sets for use in the interpolation section.
            \item Implemented the interpolation algorithms.
            \item Used the data to run tests and aggregated and analysed the results.
            \item Completed this report.
        \end{itemize}
    

\section{Mobile Sensing and Air Quality Monitoring}\label{background_mobile_sensing_models}

    Air quality, using mobility as a mechanism for data collection, can take one of four basic forms~\cite{datacollectionsurvey}. These forms are:

    \begin{itemize}
        \item Meshed sensor network
        \item Mobile sink nodes with static sensors
        \item Static sink node with mobile sensors
        \item Static sensors with data mules~\cite{datamulesthreeteir}
    \end{itemize}

    These designs are simple ideas with powerful applications. The first form, the meshed sensor network, uses constant connections with at least one other node at all times in order to create a fully connected mesh of nodes, which can route data from any point to any other. This implementation is extremely effective, as has been shown in project such as CitySense~\cite{citysense}, however this effectiveness comes at a cost. The range of radios are limited and so many sensors must be used in order to achieve a great enough density to have a connected network. The implementation is also significantly more complex than the other models as we must build in routing algorithms. 

    As a contrast, the mobile sink nodes with static sensors model keeps the sensors in one location. This model is much simpler in that a sink node moves around and collects data from the sensors it comes within a certain distance to. This distance is dictated by the wireless communication range. This model also requires less sensors as it does not need to maintain a constant connectedness. Care must be taken to ensure that the sink node visits the other nodes before they fill up their storage space and start to drop readings.

    The static sink node with mobile sensors is the inverse of the previous model. In this model we have stationary sink nodes which collect the data. The sensor nodes have mobility and as they move into range of a sink node, the data is transferred. Again, in order for this system to effectively work the sensor nodes must come into contact with a sink node within a reasonable period of time. It should be noted that we can have multiple sink nodes.

    The final model is using static sensors with data mules. This model is a variation on the previous model in that the sensors are static, however the sink also remains static. In order for data to get from the sensors to the sink a \emph{data mule} moves around the sensors collecting data, before ferrying said data to a sink node. 

    In our system we are using the static sink model with mobile sensors. The reason for this is that it allows us to exploit some of the natural mobility which exists in a city. In the case of this project this mobility is provided by buses. The sensors will be placed on these buses which will move around the city collecting data. As the buses pass open WiFi access points, i.e. the static sinks, they will upload the data. We could have used static sensors with the buses as data mules, however the buses in the city of Edinburgh provide fairly good coverage of the city with lower expenses than static sensors covering the city. 

    Currently, standard methods of collecting air quality data depend on using static sensors and then collecting data in one of two ways. The first is that the sensors are visited by a person and the data loaded onto some form of removable storage. The second, and now much more popular method, is to have these sensors connected to the internet via some mechanism as seen by organisations such as \emph{Scottish Air Quality}~\cite{scottishairquality}. However, as static sensors have much poorer coverage than mobile sensors, alternative are still required. 

    In terms of other projects which have a focus on air quality, many have adopted similar mobility strategies to the one in this project. In Zurich, the OpenSense\cite{opensensezurich} team have placed sensors on trams. Their project differs slightly from this project in that they use a cellular connection to return the data. This model more closely approximates the meshed sensor network model due to the constant connectedness. Other air quality projects with a similar model are shown by Devarakonda et al.~\cite{rtairquality} which uses cars as the vehicle and the CommonSense~\cite{commonsense} project which uses street sweepers. 


\section{Interpolation Methods}\label{background_interpolation_methods}
    This section focuses on the various statistical methods of interpolating and extrapolating data values at arbitrary points on a grid. These methods vary in their effectiveness on different data sets, however it is important to note that none of these methods take into account data which could be viewed as extremely important in this case. An example of this is the terrain. In section \todo{Fill in this reference} we discuss the effect that canyoning has on air quality readings. This effect is ignored by these models. 

    A further crucial piece of information which needs to be taken into account is that these models do not take into account temporal changes of readings. In order to use our data sets we will have to have a way of ``snapshotting'' data. A choice needs to be made for the length of time each snapshot should be. A suggestion for a rough value is 30 minutes, however experimental results will reveal the best value to use.

    With regards to the models chosen, various possibilities were considered. Due to the fact that various algorithms are similar with minor differences, the decision was made to use only a single algorithm from each ``family''. As such the algorithms chosen to be evaluated are:

    %http://en.wikipedia.org/wiki/Multivariate_interpolation
    \begin{itemize}
        \item Bicubic interpolation
        \item Natural neighbour interpolation
        \item Spline interpolation
        \item Nearest neighbour interpolation
        \item Inverse distance weighting interpolation
        \item Barnes interpolation
    \end{itemize}

    These algorithms can be split into two categories, regular grid algorithms and irregular grid algorithms, based on the format of the input data expected. 

    \subsection{Regular Grid Algorithms}\label{background_interpolation_methods_regulargrid}

        These interpolation methods require a regular grid with data points placed onto them and will calculate all missing values. In order to conform to this the data into ``buckets''. Buckets of around $5m^{2}$ will allow us to have a reasonable resolution for data without our calculations taking too long. The boxed area which the trams cover in Zurich is roughly $5km^{2}$. This gives us a grid resolution of 1,000*1,000, for a total of 800,000 data points. 

        \subsubsection{Bicubic}\label{background_interpolation_methods_bicubic}

            Bicubic interpolation is the simplest of the grid interpolation algorithms which will be evaluated. As a cubic Hermite spline~\cite{practicalguidesplines} the output is smoother than linear interpolation methods such as bilinear interpolation, as we can see in figure~\ref{fig:bicubicvsbilinear}. The bicubic algorithm achieves this by taking into account the 16 points surrounding the point to be interpolated rather than just 4. 

            \centerimage{\textwidth}{images/Bicubic_vs_Bilinear_Interpolation.png}{This figure, created by Wikipedia user \emph{Berland}, shows the difference between bicubic interpolation (left) and bilinear interpolation (right) on the same dataset.}{fig:bicubicvsbilinear}\mahesh{Should this be in italics?}

            The algorithm for bicubic interpolation is relatively simple, with the result being given by the calculation:

            \begin{align*}
                p(x,y) = \sum_{i=0}^{3}{\sum_{j=0}^{3}{a_{ij}x^{i}y^{j}}}
            \end{align*}

        %\subsubsection{B\'{e}zier Surface}\label{background_interpolation_methods_beziersurface}
        %Probably not since the surface is stretched towards the points, but does not pass through them
        
        %\subsubsection{Lanczos Resampling}\label{background_interpolation_methods_lanczosresampling}
        %Used to interpolate between a sampled signal.
        
        %\subsubsection{Delaunay Triangulation}\label{background_interpolation_methods_delaunaytriangulation}
        %Doesn't give a smooth curve


        \subsubsection{Barnes}\label{background_interpolation_methods_barnes}

            Barnes interpolation uses a multi-pass approach to determine the new data points. The method has found success in calculating air pressure across the United States, providing results similar to careful analysis, however it depends on the data points be reasonably uniform~\cite{barnesinterpolation}.

            No examples existed in either R or Python and so a custom implementation written in Python was created. This implementation follows the information in the original paper by Barnes and has shown success on test data sets. The algorithm works by calculating a simple distance weighted interpolation as the first result, and then iterating multiple times using a calculated error field to reduce the errors in the output. 

            One important factor in Barnes interpolation is the fact that it depends on server constants. These constants depend on the type of data being interpolated and the nature of the measurements, including the density of the measurements. As such, determining these constants is a key part of using this algorithm for interpolation. One advantage of this approach, is that we can iterate over our data set and fit it to the known measurements in order to make sure it is as accurate as possible. With this method however, we lose test data points and so cannot validate it. Experiments using this algorithm have used similar mechanisms and shown success~\cite{pmconcentrationmaps}.

            The algorithm for Barnes interpolation is as follows.

            For the first pass each known point is assigned a weight using the formula: 

            \begin{align*}
                W_{i} &= e^{-(d/R^{2})}
            \end{align*}
            
            where d is the distance between the known point and the current point to be interpolated, and R is the radius of influence. Using this weight, the initial guess of the grid points is calculated as: 
            
            \begin{align*}
                X_{g} &= \frac{\sum_{i}{W_{i}X_{i}}}{\sum_{i}{W_{i}}}
            \end{align*}

            At this point we begin our successive passes. These are defined as:

            \begin{align*}
                X'_{g} &= X_{g} + \frac{\sum_{i}{W'_{i}E_{i}}}{\sum_{i}{W'_{i}}}
            \end{align*}

            where $E_{k}$ is the difference between the estimated value and the actual value at a known point $k$ and $W'_{i}$ is defined as:

            \begin{align*}
                W'_{i} &= e^{-(d/\Gamma R)^{2}}
            \end{align*}

            with $\Gamma$ as a convergence parameter normally set in the range 0.2-0.3.

            The created Python code for this algorithm is as follows:

            \tdi{Fix this code and make it nice.}
            \inputminted[mathescape,linenos,numbersep=5pt,frame=lines,framesep=2mm]{python}{../Data/OpenSense/barnes.py}

        
        \subsubsection{Natural Neighbor}\label{background_interpolation_methods_naturalneighbour}
        %Seems like an interesting successor to bicubic
        
        \subsubsection{Spline Interpolation}\label{background_interpolation_methods_splineinterpolation}
        %Basic spline method

    \subsection{Irregular Grid Algorithms}\label{background_interpolation_methods_irregular_grid}

        %\subsubsection{Radial Basis Function}\label{background_interpolation_methods_radial_basis_function}
        %The value will dip between points so probably not what is needed

        %\subsubsection{Thin Plate Spline}\label{background_interpolation_methods_thin_plate_spline}
        %Seems complicated

        %\subsubsection{Least-Squares Spline}\label{background_interpolation_methods_least_squares_spline}
        %A possibility

        %http://en.wikipedia.org/wiki/Nearest-neighbor_interpolation
        \subsubsection{Nearest Neighbour}\label{background_interpolation_methods_nearest_neighbour}

            Nearest neighbour interpolation is the simplest interpolation we will see in that it is not really interpolation at all. Instead the value of each point is the value of the closest data point we have. A na\"{\i}ve version of the algorithm in Python, with \emph{known\_points} being a list of tuples containing the co-ordinates and value of each known point, is as follows\:

            \inputminted[mathescape,linenos,numbersep=5pt,frame=lines,framesep=2mm]{python}{../Data/OpenSense/nearest_neighbour.py}

        \subsubsection{Inverse Distance Weighting}\label{background_interpolation_methods_inversedistanceweighting}

            Inverse Distance Weighting, known as IDW, is the second simplest algorithm. Each point is a weighted average of all known points, where the weighting is the inverse of the distance. 

            Each value is calculated as follows:

            \begin{align*}
                V = \frac{\sum_{i=1}^{n}{\frac{v_{i}}{d^{p}_{i}}}}{\sum_{i=1}^{n}{\frac{1}{d^{p}_{i}}}}
            \end{align*}

            Once again, there are parameters which must be supplied to the algorithm which are data dependent. In this case the parameter is the smoothing factor. A lower smoothing factor means that the interpolated value at a known data point is closer to the value of the known data point. A higher smoothing factor makes the rate of change smaller to give a smoother graph and the cost of some interpolated points being different to known values which are nearby.

        %\subsubsection{Kriging}\label{background_interpolation_methods_kriging}
        % Very complicated and requires a lot of domain specific knowledge



\section{Radio Signal Propagation Models}


    When implementing a simulation of wireless communications one of the decisions which must be made is which radio propagation model to use. These propagation models define how different signals behave under different conditions and are generally given in terms of a mathematical equation. Using these models we can enter information about our radio signal, such as frequency, transmitter power and receiver power,  and gain information such as the signal to noise ratio of the signal over some distance.

    In Omnet++, which is the simulator we use, there are six different models implemented. These models are ``Free Space Model'', ``Two Ray Ground Model'', ``Rice Model'', ``Rayleigh Model'', ``Nakagami Model'' and ``Log Normal Shadowing Model''. The Free Space model is the simplest of these models and acts the same way a signal would in ``free space''. That is to say, a space where there is no interference from external signals, or the original signal through reflection, refraction or diffraction. This model is usually used as a component of other models and formulae~\cite{friis1946note}.

    The two ray ground model is only slightly more advanced, in that it takes into account a singular reflection from some boundary, usually the ground~\cite{tworaygroundmodel}.

    The log normal shadowing model builds on this. This model has two variable parameters, $\gamma$ and $\sigma$, which are used to tune the model to the environment. Other formulations have up to 5 parameters. These parameters provide information about the types of obstacles they are likely to experience and are generally provided through empirical analysis~\cite{goldhirsh1998handbook}.

    Nakagami's distribution is a variation of a gamma distribution in that we provide two parameters to this model, $m$ and $\Omega$, which control the shape of the distribution. In order to achieve a propagation model, this distribution is applied to the power level~\cite{nakagamipowerlevel}.

    The Rice model, also known as the Longley-Rice model, has origins in telecommunications. Originally designed to take into account the terrain, it can require up to 11 input parameters~\cite{ricemodel}. This model is complicated, but provides accurate information, and indeed is used by the US Federal Communications Commission~\cite{fcclongleyrice}.

    The final model is the Rayleigh model. This model works by assuming that fading happens according to a random distribution when passing through some material. This model has shown promising results in dense urban environments~\cite{rayleighmanhattan}.



