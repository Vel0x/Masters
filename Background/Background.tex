\chapter{Background and Related Work}\label{background}




\section{Previous Work}\label{background_previous_work}

    %Talk about everything. Even the stuff I didn't use
    %The AP data from Arsham

    In the first year of this project much of the initial research was done. In this section some of the key topics will be briefly discussed.

    \subsection{Air Quality}\label{background_air_quality}

A definition of air quality is a difficult problem which has not been solved. Many institutions measure and record what they refer to as ``air quality'', however the only thing these measurements have in common is that no two are the same. In order to research the area of ``air quality'' further we must have a concrete definition of what air quality is. Current definitions give a simple definition of ``air quality'' as being antonymous to that of air pollution in that higher air quality is the equivalent to low air pollution~\cite{bcaq}. This implies that not only should the metrics used for defining air quality be considered, but also those used to define air pollution.

Looking at statements and measurements by various authorities around the world \cite{epapollutants,airqualityobjectives,cleanairnavigation,naaqs,whoguidelines}, we can build a list of common pollutants which are measured:


\begin{itemize}
\item Sulphur dioxide
\item Nitrogen dioxide
\item PM10 
\item PM2.5
\end{itemize}

These chemicals can form the basis of pollutants which in turn can help define ``air quality''. By looking at other sources~\cite{meaningsofenvironmentalterms},  proposed definition of ``air quality'' is as follows:

\begin{quote}
``The current measurements of the concentrations of certain pollutants in the air relative to the requirements of one or more biotic species or to any human need or purpose.''
\end{quote}


    \subsection{Geostatistical Mapping}
    
        It was discovered that when working with an actual data set, obtained by using a sensor built by a previous years student, \emph{Eric Staples}, work would have to be performed in order to interpolate the data effectively. Generally, heat maps are used to show various concentrations and levels of variables in a two dimensional environment and so the same would be used in this project. The interpolation method was a more difficult question. Initial research suggested that some sort of convolution filter, or blur, when applied to a sufficiently dense data set would give a reasonable approximation, however this was later found to be incorrect. Further research showed that when interpolating environmental data such as pollutant levels, the most common method is kriging, a statistical method which fits a curve to the data set. Kriging is used as the basis in many advanced models. 

    \subsection{Compressive Sampling}

        Compressive sampling was looked into briefly as a method of enhancing our data collection methodology. The basis behind compressive sampling is that more measurements are taken in areas of high variance, with less measurements taken in areas of lower variance. Essentially this is a variable method of conforming to the Nyquist theorem. The work on this area was discovered to involve much background research into the nature of the pollutants and the areas they build up. It would not work if there were to be a change in the environment. As such this technique was not used in the final solution.

    \subsection{Sensor Creation}

        Initially this project was intended to have a practical component, however due to a more advanced group within the university already working with Lothian Buses, preference was given to their project. Before this change, research was done into building a physical sensor using cheap, off the shelf parts, which would be dynamically calibrated using the techniques discussed in the OpenSense project paper \cite{ontheflycalibration}. Individual sensors were researched and found, and indeed purchased in order to make a prototype, before cancelling the physical implementation. 
    


\section{Interpolation Methods}\label{background_interpolation_methods}
    This section focuses on the various statistical methods of interpolating and extrapolating data values at arbitrary points on a grid. These methods vary in their effectiveness on different data sets, however it is important to note that none of these methods take into account data which could be viewed as extremely important in this case. An example of this is the terrain. In section \todo{Fill in this reference} we discuss the effect that canyoning has on air quality readings. This effect is ignored by these models. 

    A further crucial piece of information which needs to be taken into account is that these models do not take into account temporal changes of readings. In order to use our data sets we will have to have a way of ``snapshotting'' data. A choice needs to be made for the length of time each snapshot should be. A suggestion for a rough value is 30 minutes, however experimental results will reveal the best value to use.

    With regards to the models chosen, various possibilities were considered. Due to the fact that various algorithms are similar with minor differences, the decision was made to use only a single algorithm from each ``family''. As such the algorithms chosen to be evaluated are:

    %http://en.wikipedia.org/wiki/Multivariate_interpolation
    \begin{itemize}
        \item Bicubic interpolation
        \item Natural neighbour interpolation
        \item Spline interpolation
        \item Nearest neighbour interpolation
        \item Inverse distance weighting interpolation
        \item Barnes interpolation
    \end{itemize}

    These algorithms can be split into two categories, regular grid algorithms and irregular grid algorithms, based on the format of the input data expected. 

    \subsection{Regular Grid Algorithms}\label{background_interpolation_methods_regulargrid}

        These interpolation methods require a regular grid with data points placed onto them and will calculate all missing values. In order to conform to this the data into ``buckets''. Buckets of around $5m^{2}$ will allow us to have a reasonable resolution for data without our calculations taking too long. The boxed area which the trams cover in Zurich is roughly $5km^{2}$. This gives us a grid resolution of 1,000*1,000, for a total of 800,000 data points. 

        \subsubsection{Bicubic}\label{background_interpolation_methods_bicubic}

            Bicubic interpolation is the simplest of the grid interpolation algorithms which will be evaluated. As a cubic Hermite spline~\cite{practicalguidesplines} the output is smoother than linear interpolation methods such as bilinear interpolation, as we can see in figure~\ref{fig:bicubicvsbilinear}. The bicubic algorithm achieves this by taking into account the 16 points surrounding the point to be interpolated rather than just 4. 

            \centerimage{\textwidth}{images/Bicubic_vs_Bilinear_Interpolation.png}{This figure, created by Wikipedia user \emph{Berland}, shows the difference between bicubic interpolation (left) and bilinear interpolation (right) on the same dataset.}{fig:bicubicvsbilinear}\mahesh{Should this be in italics?}

            The algorithm for bicubic interpolation is relatively simple, with the result being given by the calculation:

            \begin{align*}
                p(x,y) = \sum_{i=0}^{3}{\sum_{j=0}^{3}{a_{ij}x^{i}y^{j}}}
            \end{align*}

        %\subsubsection{B\'{e}zier Surface}\label{background_interpolation_methods_beziersurface}
        %Probably not since the surface is stretched towards the points, but does not pass through them
        
        %\subsubsection{Lanczos Resampling}\label{background_interpolation_methods_lanczosresampling}
        %Used to interpolate between a sampled signal.
        
        %\subsubsection{Delaunay Triangulation}\label{background_interpolation_methods_delaunaytriangulation}
        %Doesn't give a smooth curve


        \subsubsection{Barnes}\label{background_interpolation_methods_barnes}

            Barnes interpolation uses a multi-pass approach to determine the new data points. The method has found success in calculating air pressure across the United States, providing results similar to careful analysis, however it depends on the data points be reasonably uniform~\cite{barnesinterpolation}.

            No examples existed in either R or Python and so a custom implementation written in Python was created. This implementation follows the information in the original paper by Barnes and has shown success on test data sets. The algorithm works by calculating a simple distance weighted interpolation as the first result, and then iterating multiple times using a calculated error field to reduce the errors in the output. 

            One important factor in Barnes interpolation is the fact that it depends on server constants. These constants depend on the type of data being interpolated and the nature of the measurements, including the density of the measurements. As such, determining these constants is a key part of using this algorithm for interpolation. One advantage of this approach, is that we can iterate over our data set and fit it to the known measurements in order to make sure it is as accurate as possible. With this method however, we lose test data points and so cannot validate it. Experiments using this algorithm have used similar mechanisms and shown success~\cite{pmconcentrationmaps}.

            The algorithm for Barnes interpolation is as follows.

            For the first pass each known point is assigned a weight using the formula: 

            \begin{align*}
                W_{i} &= e^{-(d/R^{2})}
            \end{align*}
            
            where d is the distance between the known point and the current point to be interpolated, and R is the radius of influence. Using this weight, the initial guess of the grid points is calculated as: 
            
            \begin{align*}
                X_{g} &= \frac{\sum_{i}{W_{i}X_{i}}}{\sum_{i}{W_{i}}}
            \end{align*}

            At this point we begin our successive passes. These are defined as:

            \begin{align*}
                X'_{g} &= X_{g} + \frac{\sum_{i}{W'_{i}E_{i}}}{\sum_{i}{W'_{i}}}
            \end{align*}

            where $E_{k}$ is the difference between the estimated value and the actual value at a known point $k$ and $W'_{i}$ is defined as:

            \begin{align*}
                W'_{i} &= e^{-(d/\Gamma R)^{2}}
            \end{align*}

            with $\Gamma$ as a convergence parameter normally set in the range 0.2-0.3.

            The created Python code for this algorithm is as follows:

            \tdi{Fix this code and make it nice.}
            \inputminted[mathescape,linenos,numbersep=5pt,frame=lines,framesep=2mm]{python}{../Data/OpenSense/barnes.py}

        
        \subsubsection{Natural Neighbor}\label{background_interpolation_methods_naturalneighbour}
        %Seems like an interesting successor to bicubic
        
        \subsubsection{Spline Interpolation}\label{background_interpolation_methods_splineinterpolation}
        %Basic spline method

    \subsection{Irregular Grid Algorithms}\label{background_interpolation_methods_irregular_grid}

        %\subsubsection{Radial Basis Function}\label{background_interpolation_methods_radial_basis_function}
        %The value will dip between points so probably not what is needed

        %\subsubsection{Thin Plate Spline}\label{background_interpolation_methods_thin_plate_spline}
        %Seems complicated

        %\subsubsection{Least-Squares Spline}\label{background_interpolation_methods_least_squares_spline}
        %A possibility

        %http://en.wikipedia.org/wiki/Nearest-neighbor_interpolation
        \subsubsection{Nearest Neighbour}\label{background_interpolation_methods_nearest_neighbour}

            Nearest neighbour interpolation is the simplest interpolation we will see in that it is not really interpolation at all. Instead the value of each point is the value of the closest data point we have. A na\"{\i}ve version of the algorithm in Python, with \emph{known\_points} being a list of tuples containing the co-ordinates and value of each known point, is as follows\:

            \inputminted[mathescape,linenos,numbersep=5pt,frame=lines,framesep=2mm]{python}{../Data/OpenSense/nearest_neighbour.py}

        \subsubsection{Inverse Distance Weighting}\label{background_interpolation_methods_inversedistanceweighting}

            Inverse Distance Weighting, known as IDW, is the second simplest algorithm. Each point is a weighted average of all known points, where the weighting is the inverse of the distance. 

            Each value is calculated as follows:

            \begin{align*}
                V = \frac{\sum_{i=1}^{n}{\frac{v_{i}}{d^{p}_{i}}}}{\sum_{i=1}^{n}{\frac{1}{d^{p}_{i}}}}
            \end{align*}

            Once again, there are parameters which must be supplied to the algorithm which are data dependent. In this case the parameter is the smoothing factor. A lower smoothing factor means that the interpolated value at a known data point is closer to the value of the known data point. A higher smoothing factor makes the rate of change smaller to give a smoother graph and the cost of some interpolated points being different to known values which are nearby.

        %\subsubsection{Kriging}\label{background_interpolation_methods_kriging}
        % Very complicated and requires a lot of domain specific knowledge


\section{Mobile Sensing Models}\label{background_mobile_sensing_models}

\maheshi{Not sure about this section at all. In terms of mobile sensing models I could talk about some of the different experiments that have been done such as CarTel, Cafnet and OpenSense, but I would only be touching upon this again, and stating what they do?}


\section{Wireless Propagation Models}

    \maheshi{Not sure how much to write about this? It is a complex area and if I go into it any further I'd spend the remaining two weeks just reading up on this.}

    When implementing a simulation of wireless communications on of the decisions which must be made is which radio propagation model to use. These propagation models define how different signals behave under different conditions. These functions are generally given in terms of a mathematical equation. Using these models we can enter information about our radio signal, such as frequency, transmitter power and receiver power,  and gain information such as the signal to noise ratio of the signal over some distance.

    In Omnet++, which is the simulator we will use, there are six different models implemented. These models are ``FreeSpaceModel'', ``TwoRayGroundModel'', ``RiceModel'', ``RayleighModel'', ``NakagamiModel'' and ``LogNormalShadowingModel''. The Free Space model is the simplest of these models and acts the same way a signal would in ``free space''. That is to say, a space where there is no interference from external signals, or the original signal through reflection, refraction or diffraction. This model is usually used as a component of other models. 

    The two ray ground model is only slightly more advanced, in that it takes into account a singular reflections from some boundary, usually the ground.

    The log normal shadowing model builds on this again. This model has two variable parameter $\gamma$ and $\sigma$ which are used to tune the model to the environment. These parameters provide information about the types of obstacles they are likely to experience and are generally provided through empirical analysis. 

    Nakagami's distribution is a variation of a gamma distribution in that we provide two parameters to this model, $m$ and $\Omega$, which control the shape of the distribution. In order to achieve a propagation model, this distribution is applied to the power level~\cite{nakagamipowerlevel}.

    The Rice model, also known as the Longley-Rice model, has origins in telecommunications. Originally designed to take into account terrain, it requires up to 11 input parameters~\cite{ricemodel}. This model is complicated, but provides accurate information, and indeed is used by the US Federal Communications Commission~\cite{fcclongleyrice}.

    The final model is the Rayleigh model. This model works by assuming that fading happens according to a random distribution when passing through some material. This model has shown promising results in dense urban environments~\cite{rayleighmanhattan}.



