\chapter{Introduction}\label{introduction}


    %\plan[Introduction]{
    %    \begin{itemize}
    %        \item Indicate the goals of the project in crisp terms (simulation is a means)
    %            \begin{itemize}
    %                \item Performance study of an urban mobile air quality monitoring system leveraging public transport vehicles:
    %                    \begin{itemize}
    %                        \item Effects of various parameters including number of roadside access points, storage capacity (queue size) on mobile sensor platforms.
    %                        \item The benefits of opportunistic forwarding between mobiles
    %                    \end{itemize}
    %                \item Comparative evaluation of common spatial prediction methods for blanket air quality estimation in a city.
    %                \item Use Edinburgh as a case study and simulation as the methodology (specifically, OMNeT++)
    %            \end{itemize}
    %    \end{itemize}
    %}    

    %Paragraph about what is currently going on in air quality measurement
    Measuring air quality is an increasingly popular topic as the population cares more about their health. Current methods rely on static sensors which take space and can only measure pollutants in the immediate vicinity. By introducing a mobility aspect to these sensors, we can achieve a greater coverage of measurements as Aoki et al. shown in \cite{vehicleforresearch}, which used street sweepers and cellular connections in a city in the United States. Many projects rely on cellular connectivity for returning data from large mobile networks, but recent trends shown a greater push towards using routing between nodes in the network to return the data~\cite{manetmessaging,cafnet,cartel,commonsense}, no doubt due to the costs of cellular connectivity due to the need for a connection for \emph{each} sensor. As it is a relatively new area of research it remains active. 

    As it currently stands, little research has been done into both using mobility as a mechanism for air quality monitoring with a distributed network. This project will look at this area further. In terms of a case study, the City of Edinburgh Council has shown interest in being able to have more localised measurements. The success shown by the mobility model used by Aoki et al., and other similar models such as the OpenSense project~\cite{opensensezurich} in Zurich, solidified the case for adding mobility to sensors as a method of providing these localised measurements. A project of this magnitude involves many unknown parameters and this project aims to identify those. In this project we will be using Lothian Buses as the mobility mechanism for the sensors. Lothian Buses have a very respectable coverage of the city across their 70 different routes as seen in figure~\ref{fig:lothian_bus_routes}. By placing sensors on some of these buses, we can take advantage of the coverage that is offered. In a city such as Edinburgh a popular facility is open access IEEE 802.11 access points, henceforth known as wifi access points. Measurements by a PhD student named Arsham Farshad gave the location of over 2,000 wifi access points across the city which had signals strong enough to be measured from a bus. With this knowledge it is clear that we can use wifi as the delivery mechanism, thus removing the costs associated with cellular data. 

    \centerimagewideanywhere{./images/Lothian_Bus_Coverage.png}{A map of the coverage of the city of Edinburgh provided by Lothian Buses routes. This image uses \emph{Google Maps}.}{fig:lothian_bus_routes}

    With the above information in mind, the first goal of this project is to evaluate the performance of an urban air quality monitoring system, utilising the mobility of Lothian Buses, including how various parameters affect the results of the model. Due to the use of wifi we can also implement an algorithm which has shown improvement in similar situations, known as \emph{opportunistic forwarding}~\cite{opporunisticforwarding}, and evaluate the effects it has on the performance. 

    In order for this data to be of use in any scientific context we must convert the results from discrete measurements at various locations to a continuous data set. In order to do this interpolation algorithms are employed. Standard methods of interpolating air quality data include \emph{Land-Use Regression} (LUR) modelling. Methods such as LUR require domain specific knowledge such as sources of pollution and terrain information. Without access to this data, a different approach is required. This report will look at the effectiveness of using simple, well known, interpolation algorithms, including those originally designed for image and sound interpolation. 

    \tdi{What now?}%Accomplishments in terms of addressing those goals

    The remainder of this report is laid out as follows: In chapter~\ref{background} we look at background information for this project including definitions for ``air quality'', the basics of interpolation algorithms and the work that was carried out last year. Chapter~\ref{simulation} discusses the use of a simulator in realising the model as well as problems faced in selecting a simulator and implementing the simulation. Chapter~\ref{data_gathering_performance} discusses the performance of the model and how various parameter modifications change the results, including the introduction of opportunistic forwarding. The use of this data and the effectiveness of simple interpolation methods is analysed in chapter~\ref{prediction_evaluation}. Finally in chapter~\ref{conclusions} we draw the conclusions of this project.
