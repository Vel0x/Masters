\chapter{Introduction}\label{introduction}


    %\plan[Introduction]{
    %    \begin{itemize}
    %        \item Indicate the goals of the project in crisp terms (simulation is a means)
    %            \begin{itemize}
    %                \item Performance study of an urban mobile air quality monitoring system leveraging public transport vehicles:
    %                    \begin{itemize}
    %                        \item Effects of various parameters including number of roadside access points, storage capacity (queue size) on mobile sensor platforms.
    %                        \item The benefits of opportunistic forwarding between mobiles
    %                    \end{itemize}
    %                \item Comparative evaluation of common spatial prediction methods for blanket air quality estimation in a city.
    %                \item Use Edinburgh as a case study and simulation as the methodology (specifically, OMNeT++)
    %            \end{itemize}
    %    \end{itemize}
    %}    

    %Paragraph about what is currently going on in air quality measurement
    Measuring urban air quality is becoming increasingly important issue given the adverse effects of air pollution on citizens' health~\cite{beijing2point5high}. Current methods rely on sparsely distributed sets of static and expensive sensors that are not adequate for fine-grained and blanket air quality monitoring in a city. This can be remedied to a substantial extent via mobile sensors as Aoki et al. showed in \cite{vehicleforresearch}, which used street sweepers and cellular connections in a city in the United States. Many projects rely on cellular connectivity for returning data from large mobile networks, but recent trends shown a greater push towards using routing between nodes in the network to return the data~\cite{manetmessaging,cafnet,cartel,commonsense}, no doubt due to the costs of cellular connectivity and the need for a connection for \emph{each} sensor. As it is a relatively new area of research it remains active. 

    Currently, there is limited research on mobile air quality monitoring in cities. This project will look at this area considering the city of Edinburgh as a case study. The success shown by the mobility model used by Aoki et al., and other similar models such as the OpenSense project~\cite{opensensezurich} in Zurich, strengthens the case for using mobile entities to monitor urban air quality. However there are several research issues underlying this paradigm that are still not fully understood. This aims to address two such issues concerning impact of various system parameters on data gathering performance, and estimating citywide air pollution maps with measurements from limited set of locations. 

    In this project we will be using public transport buses as mobile entities on which air quality sensors are installed and use movement trace of Lothian Buses as the mobility model for the sensors. Buses have a fairly good coverage of the city across their 70 different routes, as seen in \ref{fig:lothian_bus_routes}. If air quality sensors were to be installed on some of these buses, we can expect to have spatially wider and finer measurement than with a handful of static sensors. Furthermore, in cities such as Edinburgh it is common to see a large number of \emph{open} access WiFi access points. Measurements by PhD student Arsham Farshad gave the location of over 2,000 WiFi access points (APs) across Edinburgh which had signals strong enough to be measured from a bus. Leveraging such open APs for transferring the measured sensor data from buses passing by would avoid the costs associated with cellular\footnote{Lothian Buses do provide WiFi on select routes, but currently this is not wide spread and is not as seamless in its use as one must register an account and their device before use.} data connections. 

    \centerimagewideanywhere{./images/Lothian_Bus_Coverage.png}{A map of the coverage of the city of Edinburgh provided by Lothian Buses routes. The coloured lines represent the routes and the black circles are bus stops. This image uses \emph{Google Maps}.}{fig:lothian_bus_routes}\todo{Add the locations of the SAQ stations}

    The first goal of this project is to evaluate the impact of various system parameters (e.g., number of APs, buffer capacity) on the data gathering performance of a bus-based city air quality monitoring system via simulation and using the above mentioned mobility model. We also explore the value of using other buses as intermediate relays for uploading the sensor data to roadside APs. Specifically, we evaluate the performance of a simple relaying method we refer to as \emph{opportunistic forwarding}~\cite{opportunisticforwarding}. With respect to this first project goal, our findings are: 

    \tdi{Fill this in}
    \begin{itemize}
        \item First finding
        \item second finding
    \end{itemize}


    Even with mobile air quality monitoring, not every location in the city may have sensor measurements. This can be easily observed from Figure~\ref{fig:lothian_bus_routes}. So our second goal in this project is to assess the effectiveness of simple and well-known spatial interpolation algorithms for predicting air quality at locations without any measurements. Note that standard methods for spatial prediction of air quality include Land-Use Regression (LUR) but such methods require domain specific knowledge such as sources of pollution and terrain information. Our aim is to see if simpler algorithms provide reasonably accurate predictions. Our findings and observations with respect to this goal are:

    \tdi{Fill this in}
    \begin{itemize}
        \item First finding
        \item second finding
    \end{itemize}

    The remainder of this report is laid out as follows: In chapter~\ref{background} we look at background information for this project including definitions for ``air quality'', the basics of interpolation algorithms and the work that was carried out last year. Chapter~\ref{simulation} discusses the use of a simulator in realising the model as well as problems faced in selecting a simulator and implementing the simulation. Chapter~\ref{data_gathering_performance} discusses the performance of the model and how various parameter modifications change the results, including the introduction of opportunistic forwarding. The use of this data and the effectiveness of simple interpolation methods is analysed in chapter~\ref{prediction_evaluation}. Finally, in chapter~\ref{conclusions} we draw the conclusions of this project and make recommendations.
