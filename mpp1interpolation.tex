\section{Interpolation and Extrapolation Models}\label{interpolation}


When measuring air quality values are only generally generated at points where sensors can easily be placed. In order to effectively map air pollution a method of calculating the missing values must be found. There are many different models which can be used to calculate missing values.


	\begin{itemize}
		\item Interpolation and Extrapolation (mainly a question of considering different schemes and models and talking about their effectiveness)
		\begin{itemize}
			\item Discuss state of the art
			\item Comparison and contrasting
		\end{itemize}
	    \item A few paragraphs about each one
	    \item Are things like LUR relevant when I have more dense sampling of the space
	    \item Find out which model is the best for calculating air quality
	    \begin{itemize}
	        \item Implement several different models and see which is the most efficient by validating
	        \item Find the pros and cons of each one (such as statistical models never being applied to these problems)
	    \end{itemize}
	    \item Compare and contrast (such as which schemes have these been used in)
	    \begin{itemize}
	        \item Talk about the constraints of each one
	    \end{itemize}
	\end{itemize}

\subsection{Land-Use Regression Model (LUR)}\label{lur}

Land-use Regression models are they standard approach in determining missing date in air quality measurements. LUR uses the concentration measure of the pollutant, location information, and information about the area (such as is it a built up area, the altitude, are there factories nearby, etc.). 


\subsection{Geostatistical Mapping}\label{geostatics}

%http://systems.cs.colorado.edu/~caleb/dyspan2012.pdf

\subsection{Inverse Distance Weighting}\label{inverseweight}

%http://www.alyrica.net/wifi_mapping

\subsection{Comparison}\label{interpolationcomparison}

