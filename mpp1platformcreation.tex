\subsection{Platform Creation}\label{platformcreation}

\subsubsection{Sensor}\label{sensor}

Due to the great expense of using the \emph{Waspmote} platform developed by \emph{Libelium} ~\cite{waspmote}, it was decided that the most suitable option for creating a sensor platform was to use an \emph{Arduino} board of some form. Research showed that it had been used successfully in many hobbyist projects and also in other sensor network research projects. ~\cite{arduinoproj1}~\cite{arduinoproj2}~\cite{arduinoproj3} 

The \emph{Arduino Uno} was selected as a suitably powered, low cost system for the development base. Extra components provided wifi and GPS along with sensors for temperature, humidity, carbon monoxide and ozone. 

Due to a concern about processing power by Dr Marina, a \emph{Raspberry Pi} was also purchased to develop a base sensor from. Should the Uno prove unsuitable the Raspberry Pi will be used. In order to facilitate a faster development time, a conversion shield from Arduino Uno to Raspberry Pi was also purchased. This allows the physical set up of the sensors to stay constant between the two systems. The \emph{arduPi} library ~\cite{ardupi} created by Libelium also allows the software created for the Uno to run on the Pi. 

As the sensor will be subject to a wide variety of weather conditions, a suitable enclosure must be found. This enclosure should protect the electronic components while still allowing readings to be taken. There should also be the ability to provide power to the sensor somehow. This enclosure has not yet been decided on as it is dependent on where and how the sensor will be mounted. 

Currently the components have not yet been delivered and therefore the sensor has not been implemented. 

\subsubsection{Data Collection Framework}\label{datacollectionframework}

In January of this year, preliminary readings were taken with a sensor designed and developed last year by Eric Shane Staples.~\cite{envisensor} This preliminary work was designed as a test for multiple concepts. One of these was a stable data recording and uploading protocol. The sensor would take readings and send them to the receiver over bluetooth which would attempt to upload directly to a server by performing a GET request containing the data to a server. If this failed for whatever reason, or the application detected that there was no internet connection before attempting this, the data would be cached to be uploaded at a later date. The trial was not very demanding but showed that the concept worked in principle. The system can easily be scaled to deal with much more data being returned from any upload points. 

The second part to the framework is the data visualisation ability. The work into heat maps and interpolation models will allow the data to be viewed in simple graphical formats, while still retaining more complex charts and tables. After discussion with Dr Mark Wright, an expert in human-computer interaction, he has agreed to provide his expertise in designing an easy to use, clear interface should it be required. 
