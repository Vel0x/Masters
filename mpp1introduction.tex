\section{Introduction}\label{intro}


Air quality, while not always noticeable, is a very important environmental measurement which should not be ignored. Air pollution levels are rising in many places across the planet, with cities such as Beijing having levels as high as 700 $\upmu$gm$^{-3}$ of PM2.5\todo{CITE ME}. In order to combat these life endangering numbers, we must first have accurate measurements of the pollutants. In Edinburgh, the levels of PM2.5 appear to hover around 8-10 $\upmu$gm$^{-3}$, occasionally spiking up towards 40-50 $\upmu$gm$^{-3}$ ~\cite{pm2point5inscotland} and so is generally much less of a problem. This decrease in Edinburgh is similar with most other air pollutants, however Edinburgh can still be used to test and evaluate pollutant measuring and collection models. 

The aim of the project is to evaluate the suitability of the following model \todo{What model?} of taking air quality measurements with the goal of providing data quickly, whilst still remaining efficient. 


\todo[inline]{Need to talk about buses}
